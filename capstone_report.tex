\documentclass[]{article}
\usepackage{lmodern}
\usepackage{amssymb,amsmath}
\usepackage{ifxetex,ifluatex}
\usepackage{fixltx2e} % provides \textsubscript
\ifnum 0\ifxetex 1\fi\ifluatex 1\fi=0 % if pdftex
  \usepackage[T1]{fontenc}
  \usepackage[utf8]{inputenc}
\else % if luatex or xelatex
  \ifxetex
    \usepackage{mathspec}
    \usepackage{xltxtra,xunicode}
  \else
    \usepackage{fontspec}
  \fi
  \defaultfontfeatures{Mapping=tex-text,Scale=MatchLowercase}
  \newcommand{\euro}{€}
\fi
% use upquote if available, for straight quotes in verbatim environments
\IfFileExists{upquote.sty}{\usepackage{upquote}}{}
% use microtype if available
\IfFileExists{microtype.sty}{%
\usepackage{microtype}
\UseMicrotypeSet[protrusion]{basicmath} % disable protrusion for tt fonts
}{}
\usepackage[margin=1in]{geometry}
\usepackage{graphicx}
\makeatletter
\def\maxwidth{\ifdim\Gin@nat@width>\linewidth\linewidth\else\Gin@nat@width\fi}
\def\maxheight{\ifdim\Gin@nat@height>\textheight\textheight\else\Gin@nat@height\fi}
\makeatother
% Scale images if necessary, so that they will not overflow the page
% margins by default, and it is still possible to overwrite the defaults
% using explicit options in \includegraphics[width, height, ...]{}
\setkeys{Gin}{width=\maxwidth,height=\maxheight,keepaspectratio}
\ifxetex
  \usepackage[setpagesize=false, % page size defined by xetex
              unicode=false, % unicode breaks when used with xetex
              xetex]{hyperref}
\else
  \usepackage[unicode=true]{hyperref}
\fi
\hypersetup{breaklinks=true,
            bookmarks=true,
            pdfauthor={},
            pdftitle={Predicting Restaurant Failure Using Yelp Data},
            colorlinks=true,
            citecolor=blue,
            urlcolor=blue,
            linkcolor=magenta,
            pdfborder={0 0 0}}
\urlstyle{same}  % don't use monospace font for urls
\setlength{\parindent}{0pt}
\setlength{\parskip}{6pt plus 2pt minus 1pt}
\setlength{\emergencystretch}{3em}  % prevent overfull lines
\setcounter{secnumdepth}{0}

%%% Use protect on footnotes to avoid problems with footnotes in titles
\let\rmarkdownfootnote\footnote%
\def\footnote{\protect\rmarkdownfootnote}

%%% Change title format to be more compact
\usepackage{titling}

% Create subtitle command for use in maketitle
\newcommand{\subtitle}[1]{
  \posttitle{
    \begin{center}\large#1\end{center}
    }
}

\setlength{\droptitle}{-2em}
  \title{Predicting Restaurant Failure Using Yelp Data}
  \pretitle{\vspace{\droptitle}\centering\huge}
  \posttitle{\par}
  \author{}
  \preauthor{}\postauthor{}
  \predate{\centering\large\emph}
  \postdate{\par}
  \date{November 17, 2015}



\begin{document}

\maketitle


\subsection{Introduction}\label{introduction}

Restaurants have one of the highest business failure rates of all the
retail and service industries, with a 30\% failure rate commonly
accepted as the norm. Some of the macro factors associated with
restaurant failure include the economy, federal and local legislation,
climate and natural events, regional and urban planning, changing
cultural factors, and new competition. There are also many
business-related micro factors associated with restaurant failure,
including lack of capital, lack of industry experience, poor leadership,
lack of cost controls, and high fixed costs. Other micro factors that
contribute to restaurant failure include location (not just the physical
site, but the demographics of the surrounding area as well), failing to
follow the 12 Ps of restaurant branding (Place, Product, Price, People,
Promotion, Promise, Principles, Props, Production, Performance,
Positioning and Press), and even the name of the restaurant (restaurants
with names that are brief, descriptive and attractive are more likely to
succeed). For a more detailed discussion, please see
\url{http://hospitality.ucf.edu/files/2011/08/DPI-Why-Restaurants-Fail.pdf}

The primary question of interest for this project is to attempt to
identify drivers of restaurant failure (i.e., predictors of restaurants
that are no longer in business) using Yelp data. Given that Yelp data
does not contain strong predictors such as economic information, the
goal is to build a model that predicts restaurant failure using only
business and review data.

\subsection{Methods}\label{methods}

\subsubsection{Understanding the Problem and the
Dataset}\label{understanding-the-problem-and-the-dataset}

The data for this project was part of Round 6 of the
\href{http://www.yelp.com/dataset_challenge}{Yelp Dataset Challenge},
which consists of information about businesses from 10 cities in 4
countries and includes:

\begin{itemize}
\itemsep1pt\parskip0pt\parsep0pt
\item
  Over 1.5M reviews by 366K users for 61K businesses, and
\item
  Over 481K business attributes such as hours, parking, ambience, etc.
\end{itemize}

Of the five files included in the Yelp dataset (business, checkins,
reviews, tips, and users), only the business and review data files will
be used in this project. For the purposes of predicting restaurant
failure, the target variable will be whether or not a restaurant is open
for business (coded `Yes' for not in business and `No' for in business
so the model will predict failure rather than success), and the
predictor variables will be a combination of business attributes, review
data, and feature-engineered variables.

\subsubsection{Pre-Processing the Data}\label{pre-processing-the-data}

A total of 21,799 restaurants and 990,627 restaurant reviews were
extracted from the business and review data files for further
processing, based on the presence of the word `Restaurant' in the
business category variable.
\href{https://en.wikipedia.org/wiki/Types_of_restaurant}{Four restaurant
types} are represented in the data set: fast food, fast casual, casual,
and fine dining. Variables with greater than 50\% missing values were
removed from the dataset, and all missing values in the remaining
variables were recoded to 0. All character variables were then recoded
to dummy variables with the \texttt{dummyVars} function in the
\texttt{caret} package using the \texttt{fullRank = TRUE} option to
avoid creating linear dependencies.

\subsubsection{Feature Engineering}\label{feature-engineering}

Seven variables were feature-engineered for the predictive model. First,
a total of 80 different restaurant categories were identified in the
business category variable (from Afghan to Vietnamese). Second, a new
city variable identifying the 10 cities in the dataset (Canada:
Montreal, Waterloo; Germany: Karlsruhe; US: Charlotte, Las Vegas,
Madison, Phoenix, Pittsburgh, Urbana-Champaign; UK: Edinburgh) was
created by performing a \emph{k}-means cluster analysis of the
restaurants' latitude and longitude. These two variables were then
converted to continuous variables using the
\href{http://www.sciencedirect.com/science/article/pii/S0167923615000275}{Weight
of Evidence (WOE) transformation}. For the binary classification problem
in this project, WOE can be defined as
\(WOE^X_i = ln (C^X_i / TC) / (N^X_i / TN))\) where \(TC\) and \(TN\)
are the total number of closed and open restaurants, respectively, and
\(C^X_i\) and \(N^X_i\) are the number of closed and open restaurants
for the \(i\)th value of attribute \(X\).

Third, a sentiment analysis of the review verbatims was performed based
on Jeffrey Breen's
\href{https://github.com/jeffreybreen/twitter-sentiment-analysis-tutorial-201107}{Twitter
sentiment analysis tutorial} using Hu \& Liu's
\href{https://www.cs.uic.edu/~liub/FBS/sentiment-analysis.html\#lexicon}{opinion
lexicon}, where an overall sentiment score is computed based on the
number of positive words minus the number of negative words. A
normalized mean sentiment score was then computed for each restaurant by
dividing the sentiment score by the review length and taking the mean
across all reviews. The remaining feature-engineered variables for each
restaurant were mean review length (number of words), restaurant name
length (number of words), percent of review verbatims with the word
`manager' or `management' (hypothesized to be an indicator of poor
service), and percent of one-star reviews.

\subsubsection{Selecting the Modeling
Algorithm}\label{selecting-the-modeling-algorithm}

A gradient boosted decision tree classification model was selected as
the modeling algorithm for this project. Gradient boosted decision trees
fit many large or small trees to reweighted versions of the training
data, and classify on the target variable by weighted majority vote.
Tianqi Chen's eXtreme Gradient Boosting
\href{https://github.com/dmlc/xgboost}{(\texttt{XGBoost})} is a fast and
efficient implementation of
\href{https://en.wikipedia.org/wiki/Gradient_boosting}{gradient
boosting} framework. \texttt{XGBoost} yields accurate predictions for
most data sets, as evidenced in its use by several recent
\href{https://github.com/dmlc/xgboost\#whats-new}{Kaggle competition
winners}.

\subsubsection{Parameter Tuning Through
Cross-Validation}\label{parameter-tuning-through-cross-validation}

First, the modeling dataset was split 60/20/20 into training, test, and
validation sets using the \texttt{createDataPartition} function in the
\texttt{caret} package in order to preserve the overall class
distribution of the data. Next, a gradient boosted decision tree
classification model was fit to the training data using the
\texttt{xgboost} package with 10-fold cross validation repeated 5 times.
The final parameters for the optimal model were \texttt{nrounds} = 150,
\texttt{max\_depth} = 3, and \texttt{eta} = 0.3. Predictions were
computed on the test set and a confusion matrix was generated for model
evaluation. A second set of predictions was then computed on the
validation set.

\subsubsection{Building the Model}\label{building-the-model}

In order to account for class imbalance in the dataset (i.e., the low
proportion of restaurants that failed compared to those that did not),
an alternate threshold for the predicted probability was computed using
\href{http://www.researchgate.net/publication/8308935_Youden_WJIndex_for_rating_diagnostic_tests._Cancer_3\%281\%29_32-35}{Youden's
\emph{J} index}, which measures the proportions of correctly predicted
samples for both the event and nonevent groups and can be defined as
\(J = Sensitivity + Specificity - 1\). A new set of predictions was then
computed using the alternate threshold and a confusion matrix was
generated to evaluate the final model.

\subsection{Results}\label{results}

\subsubsection{Exploratory Data
Analysis}\label{exploratory-data-analysis}

The overall prevalence of restaurant failure in the Yelp dataset was
19.8\%, but varies widely by city, with only 2.1\% failure in Karlsruhe
but 24.6\% failure in Urbana-Champaign. As a result, city should be a
important predictor in the final model.

\begin{verbatim}
##        
## Failed? Charlotte Edinburgh Karlsruhe Las Vegas Madison Montreal Phoenix
##     Yes     0.185     0.142     0.021     0.226   0.215    0.093   0.236
##     No      0.815     0.858     0.979     0.774   0.785    0.907   0.764
##        
## Failed? Pittsburgh Urbana-Champaign Waterloo
##     Yes      0.180            0.246    0.066
##     No       0.820            0.754    0.934
\end{verbatim}

Restaurant failure also varies widely by restaurant category, ranging
from 0\% for bistros to 58.1\% for Cuban restaurants. The top five
restaurant categories with the highest failure percentage are shown
below. Restaurant category should also be an important predictor in the
final model.

\begin{verbatim}
##     Category Failed?
## 45     Cuban   0.581
## 48   Russian   0.467
## 39     Cajun   0.422
## 47 Soul Food   0.419
## 2      Irish   0.400
\end{verbatim}

Finally, there does appear to be a linear relationship between name
length and restaurant failure: restaurants with 6 or more words in the
name have a failure rate nearly twice as high as those with only
one-word names (25.8\% vs.~13.6\%).

\begin{verbatim}
##        
## Failed?     1     2     3     4     5     6
##     Yes 0.136 0.192 0.214 0.210 0.258 0.258
##     No  0.864 0.808 0.786 0.790 0.742 0.742
\end{verbatim}

\subsubsection{Model Building}\label{model-building}

The confusion matrix and evaluation diagnostics for the predictions of
the tuned model on the test set are shown below. The overall accuracy
was 85.3\%, which is slightly better than the no-information rate of
80.2\%. A kappa value of 0.457 suggests moderate agreement. However, the
model has low sensitivity (43.2\%), which suggests that the model has
trouble predicting restaurant failure due to the class imbalance of the
target variable and lack of strong predictors in the model.

\begin{verbatim}
##           Reference
## Prediction  Yes   No
##        Yes  373  149
##        No   491 3347
\end{verbatim}

\begin{verbatim}
##     Accuracy        Kappa AccuracyNull 
##    0.8532110    0.4572203    0.8018349
\end{verbatim}

\begin{verbatim}
## Sensitivity Specificity 
##   0.4317130   0.9573799
\end{verbatim}

While the issue of strong predictors cannot be addressed with the data
on hand, the class imbalance can be addressed by adjusting the cutoff
value for the predicted probabilities, which is set to 50\% by default.
Using an alternate cutoff of 20.5\% (i.e., probabilities greater than
0.205 are called events) based on Youden's \emph{J} index increases the
sensitivity of the model from 43.1\% to 76.5\%.

\begin{verbatim}
##   threshold specificity sensitivity 
##   0.2053138   0.7854691   0.7659328
\end{verbatim}

The confusion matrix for the model predictions on the validation set
using the alternate threshold is shown below. The final model clearly
does a much better of job of predicting restaurant failure while
achieving a better balance between sensitivity and specificity.

\begin{verbatim}
##           Reference
## Prediction  Yes   No
##        Yes  661  750
##        No   202 2746
\end{verbatim}

A dotplot of variable importance in the final model is shown below. The
top 5 predictors are review count, review length, city, restaurant
category, and the business attribute `Good For Dinner'.

\includegraphics{capstone_report_files/figure-latex/unnamed-chunk-7-1.pdf}

Restaurants that are no longer in business tend to have fewer but longer
reviews on average.

\begin{verbatim}
##   Failed? Review.Count Review.Length
## 1     Yes     26.37069      704.6122
## 2      No     55.45143      595.0459
\end{verbatim}

Restaurants that are flagged as `Good For Dinner' have a failure rate
nearly twice as high compared to those that are not (28.1\% vs 17.0\%).

\begin{verbatim}
##        
## Failed?     0     1
##     Yes 0.170 0.281
##     No  0.830 0.719
\end{verbatim}

\subsection{Discussion}\label{discussion}

The primary question of interest for this project was to identify
drivers of restaurant failure using Yelp review data. Using a powerful
prediction algorithm in the form of gradient boosted decision trees
resulted in a predictive model with reasonably high sensitivity after
the model was adjusted for class imbalance. As a result, it was possible
to answer the primary question of interest despite the lack of strong
predictors in the dataset. It is interesting to note that five of the
seven feature-engineered variables were among the top 10 drivers in
terms of variable importance. Without feature engineering it is doubtful
that an accurate predictive model could have been achieved.

Review count and review length as the top two drivers of restaurant
failure was unexpected. Fewer reviews on average for failed restaurants
is perhaps not too surprising, as review count could be considered to be
a proxy for popularity. However, the finding that reviews for failed
restaurants are over 100 words longer on average compared to reviews of
successful restaurants suggests that longer reviews tend to express
concern rather than praise.

The city and restaurant category variables played an important role in
predicting restaurant failure, as initially hypothesized. However, the
prevalence of restaurant failure in the Canadian and German cities was
much lower than the other cities in the dataset, and not even close to
the 30\% rate considered to be the industry norm. As a result, while
city was one of the top five drivers of restaurant failure identified by
the model, removing these three cities from the model would likely have
resulted in a much lower importance score for the city variable. The
restaurant category variable suggests that restaurants with
non-traditional cuisines such as Cuban, Russian, Cajun and Soul Food run
the highest risk of failure. The `Good for Dinner' flag suggests that
restaurants at the higher end of the price scale (i.e., casual and fine
dining) run the highest risk of failure.

In terms of further analysis, it would be interesting to see if the
relationship between business failure, review count and review length
identified in this project generalizes to other businesses in other
cities. While an analysis of Yelp reviewers themselves was not
considered in this project, a follow-up analysis of review length and
rating by reviewer would also be an interesting analysis.

\end{document}
